\documentclass[conference]{IEEEtran}
\IEEEoverridecommandlockouts
% The preceding line is only needed to identify funding in the first footnote. If that is unneeded, please comment it out.
\usepackage{cite}
\usepackage{amsmath,amssymb,amsfonts}
\usepackage{algorithmic}
\usepackage{graphicx}
\usepackage{textcomp}
\usepackage{xcolor}
\usepackage{hyperref}
\hypersetup{
    colorlinks=true,
    linkcolor=blue,
    filecolor=magenta,      
    urlcolor=cyan,
}
\def\BibTeX{{\rm B\kern-.05em{\sc i\kern-.025em b}\kern-.08em
    T\kern-.1667em\lower.7ex\hbox{E}\kern-.125emX}}
\begin{document}

\title{Wayne Havey Journal \#3\\
}

\author{\IEEEauthorblockN{1\textsuperscript{st} Wayne R. Havey III}
\IEEEauthorblockA{\textit{PhD Security} \\
\textit{UCCS}\\
Colorado Springs, CO \\
whavey@uccs.edu}
}

\maketitle

\section{This Weeks Learning: Survey Paper}
This week Ive been learning to refine the browse process for identifying potential survey papers. 
\vspace{5mm}

First of all I learned that my potential field of study is fairly saturated but still new enough that there is likely a space for me to find some room to work in. This is about what I expected. Nailing down some keywords to use in a google scholar search took only minimal time and I found the more papers I found from one keyword the more keywords I could find to perform more searches and so on. I discovered I could have added papers with survey potential to my library indefinitely so I decided to limit myself to 90; 3 times the amount of references I need. I may need to go back through another pass. The tactic of looking at papers citing survey papers proved fairly effective. I was a little disheartened by the amount of survey papers in my field of interest that were less than two years old. 
\vspace{5mm}

As far as the actual browsing, I found that limiting the browse to abstracts only was effective enough to get a very high level of the "what" of the paper to be able to take some rudimentary notes. My method of note taking was to formulate in a few words a sort of category that could be applied to the paper. This way I was able to find overlap and subcategories between all the papers. I hope to look back through this high level categorization to determine a novel, or at least beneficial mapping that can be used for my survey paper. Additionally, I hope to be able architect this mapping so that there are nodes/dimensions that could be applied that few to no papers would fall into so as to find that "white space" for my own personal research in the future. 

\end{document}
